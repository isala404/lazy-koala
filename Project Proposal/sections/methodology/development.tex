{\let\clearpage\relax \chapter{Development Methodology}}

Even though this project has few clearly defined requirements, designing and developing them will require an iterative model as there isn't a single best way to develop this and the author will be experimenting with different techniques. So author decides on using \textbf{prototyping} as the Software Development Life Cycle Model for this project.

\section{Design Methodology}

To complete this project 2 programming languages need to be used, Python and GoLang. To clean up the dataset and create the model itself Python will be used because most data science tools like TensorFlow and PyTorch were written targeting Python. When it comes to the Kubernetes side, the entire Kubernetes ecosystem was developed using GoLang. Since GoLang doesn't support classes and Python isn't a very OOP-friendly language this project will follow a functional programming paradigm to develop the code-base. 

\section{Evaluation Methodology}

During the literature, the survey author concluded there ain't any specific evaluation metrics for root cause analysis system other than accuracy and f1 score and there ain't any publicly available dataset or a system to benchmark against. But with this research author is hoping to develop a benchmarking system so future researchers can benefit from this.