
{\let\clearpage\relax \chapter{Project Management Methodology}}

\section{Schedule}

\section{Deliverables}


\begin{longtable}{|p{11cm}|p{3.5cm}|}
\hline
\textbf{Deliverable} & \textbf{Date} \\ \hline
\textbf{Draft Project Proposal} & \multirow{2}{*}{02nd September 2021} \\
A draft version of this proposal &  \\ \hline
\textbf{A working beta of MicroSim}\label{microsim} & \multirow{2}{*}{15th September 2021} \\
MicroSim is a tool that simulates a distributed system within a Kubernetes cluster. This tool will be used to test and evaluate the final version of this project &  \\ \hline
\textbf{Research Paper about MircoSim} & \multirow{2}{*}{16th October 2021} \\
MicroSim could have various other use-cases and could help in the development of this research domain. So the author is planning to release it as an open-source project with paper so future research and benefits from this. &  \\ \hline
\textbf{Literature Review Document} & \multirow{2}{*}{21st October 2021} \\
The Document explaining all the existing tools and published researches on the domain &  \\ \hline
\textbf{Project Proposal} & \multirow{2}{*}{04th November 2021} \\
The final version of this project proposal. &  \\ \hline
\textbf{Software Requirement Specification} & \multirow{2}{*}{25th November 2021} \\
The Document all the key requirements that are gonna get address with this research &  \\ \hline
\textbf{Proof of Concept} & \multirow{2}{*}{06th December 2021} \\
Unoptimized prototype with all the main features working &  \\ \hline
\textbf{Interim Progress Report (IPR)} & \multirow{2}{*}{27th January 2022} \\
The document explaining all the preliminary findings and the current state of the project &  \\ \hline
\textbf{Test and Evaluation Report} & \multirow{2}{*}{17th March 2022} \\
A document with results of the project and conclusion made from those tests &  \\ \hline
\textbf{Draft Project Reports} & \multirow{2}{*}{31st March 2022} \\
The draft version of the final thesis &  \\ \hline
\textbf{Final Research Paper} & \multirow{2}{*}{14th April 2022} \\
A paper with results about this project &  \\ \hline
\textbf{Final Project Report} & \multirow{2}{*}{28th April 2022} \\
Finalize version of the thesis &  \\ \hline
\caption{Deliverables and Dates}
\end{longtable}


\section{Resource Requirement}

\subsection{Software Requirements}

\begin{itemize}
\item \textbf{Linux based operating system} - During this project author will be working with the Kubernetes ecosystem. Most of the tooling for Kubernetes like docker works best Linux kernel.
\item \textbf{Python} - Most of the tooling related to data science like TensorFlow and PyTorch are written as python libraries. Python also provides a lot of help built-in functions to  accelerate the prototyping process.
\item \textbf{GoLang} - Since Kubernetes itself is built using GoLang, It has a very mature client library with a lot of documentation to deal with Kubernetes API.
\item \textbf{Docker and K3D} - To create a Kubernetes cluster locally for development and testing.
\item \textbf{PyCharm and GoLand} - These two are the best IDEs in the market when writing Python and Go projects and it could improve the productivity of developers by big merging. As students, we get a license to use both of those IDEs for free.
\item \textbf{Overleaf / LateX}  - Managing a big document in both MS Word and Google Docs could be a cumbersome task. Especially when adding and removing things both of these software tends to mess up the layout. By using LateX we can declaratively define the layout so it will always behave as intended.
\item \textbf{Google Drive and Github} - Offsite location to backup the codebase and related documents
\item \textbf{ClickUp} - To manage the project and keep track of things to be done
\end{itemize}

\subsection{Hardware Requirements}
\begin{itemize}
    \item \textbf{A Quad-core CPU with AVX support} - Simulating a distributed system locally will take a lot of processing power. Having an AVX supported CPU will reduce the inference time when testing it on a cluster.
    \item \textbf{GPU with CUDA support and 2GB of VRAM} - Both Tensorflow and Pytorch depend on CUDA for hardware-accelerated training. Training on GPU could save a lot of time increases the number of trial and error iterations that could be done. Having more VRAM could help with building larger models.
    \item \textbf{16 GB Memory} - Running a microservices simulation locally will consume a lot of memory and while testing models will get loaded into RAM. Here having dual-channel memory will be preferable. 
    \item \textbf{20GB disk space} - Models and datasets won't take a lot of disk space but again running the Kubernetes cluster demands a lot of disk space because it needs to keep a lot of containers locally cached.
\end{itemize}

\subsection{Skill Requirements}
\begin{itemize}
    \item \textbf{Experience working with Kubernetes} - The author will be developing a Kubernetes extension so they need to know the inner workings of Kubernetes.
    \item \textbf{Data engineering} -  Developing a data encoding technique requires a lot of knowledge in how to manipulate a given dataset.
    \item \textbf{Model engineering} - To complete the project, a completely new model has to be built from the ground up. So the author needs to have an in-depth idea about how to create a model in a machine learning framework and how different layers in the model work to fit them properly. 
\end{itemize}

\subsection{Data Requirements}
\begin{itemize}
\item \textbf{Monitoring dataset} -  This dataset can be collected using \hyperref[microsim]{MicroSim} tool author plan to develop to simulate distributed system.
\end{itemize}

\section{Risk Management}


\begin{longtable}{|p{4cm}|p{2cm}|p{2cm}|p{7cm}|}
    \hline
    \textbf{Risk Item} & 
    \textbf{Severity} & 
    \textbf{Frequency} & 
    \textbf{Mitigation Plan}
    \\ \hline
    
    The hypothesis the research is based on is wrong & 
    5 & 
    1 & 
    Present the findings and explain why the hypothesis was wrong 
    \\ \hline
    
    Failure in work computer & 
    4 & 
    3 & 
    Daily backup work the work to a cloud platform 
    \\ \hline
    
    Lack of domain knowledge & 
    2 & 
    3 & 
    Talk to a domain expert, Do more research 
    \\ \hline
    
    Models not generalizing & 
    3 & 
    4 & 
    Explore different methods, Try cleaning up the dataset more 
    \\ \hline
    
    Dataset quality is not up to the standard & 
    4 & 
    1 & 
    Use a method used in related researches to create a new dataset 
    \\ \hline
    
    Running out of time & 
    1 & 
    2 & 
    Have a proper work schedule from the start 
    \\ \hline
    
    Getting sick and unable to work for few days & 
    3 & 
    3 & 
    Keeping few days of a buffer period before deadlines 
    \\ \hline
    \caption{Risks and Mitigation}
\end{longtable}