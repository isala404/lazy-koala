\section{System Architecture}

System architecture design gives a bird's-eye view of how all the components in the system communicate with each other. This helps us to understand the dependencies and responsibilities of each component. Since this system is designed to run on a microservices-based environment, an n-tier design architecture was used to physically separate the components in the system to have better a reliability and scalability.

\begin{figure}[H]
    \includegraphics[width=10cm]{assets/system-design/tier-architecture.png}
    \caption{Tiered architecture (self-composed))}
    \label{fig:tier-architecture}
\end{figure}

\subsection{Presentation Tier}

The presentation tier will be entirely running in the client's computer while depending on the logic tier for data.

\begin{itemize}[noitemsep,nolistsep] 
    \item \textbf{Monitoring Dashboard} - This view is responsible for helping the user understand the service topology and visually identify issues in the system.
    \item \textbf{Settings View} - On the settings page users can choose which services are needed to be monitored along with their DNS address.
\end{itemize}

\subsection{Logic Tier}

The logic tier will contain three custom microservices that depend on Kubernetes's core modules to operate.

\begin{itemize}[noitemsep,nolistsep] 
    \item \textbf{\ac{lazy-koala-operator}} - The \ac{lazy-koala-operator} is the main bridge between Kubernetes APIs and this system. It also contains a proxy server that redirects incoming client requests to kube-api.
    \item \textbf{\ac{gazer}} - An instance of \ac{gazer} will be running on every node in the Kubernetes cluster which passively extracts telemetry and sends them over to the Prometheus server.
    \item \textbf{\ac{sherlock}} - The AI engine periodically queries Prometheus to get the current status of all the monitored services. Then it calculates an anomaly score for each service and pushes it to Prometheus so it can be sent back to the presentation layer
    \item \textbf{kube-apiserver} - This is an API provided by Kubernetes that help to read and update the cluster status programmatically.
    \item \textbf{kube-scheduler} - kube-scheduler is responsible for smartly provision requested resources in available spaces.
    \item \textbf{kube-controller-manager} - This service sends updates to all operators running in the cluster whenever there is a change to a resource that was owned by the specific operator.
\end{itemize}

\subsection{Data Tier}

\begin{itemize}[noitemsep,nolistsep] 
    \item \textbf{Blob Storage} - Storage for the pre-trained models and built containers.
    \item \textbf{Prometheus} - A time series database that is highly optimised for telemetry collection.
    \item \textbf{etcd} - etcd is an in-memory database that will be responsible for holding the resources specifications and \ac{gazer} config.
\end{itemize}