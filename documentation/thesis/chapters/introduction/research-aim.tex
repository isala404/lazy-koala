\section{Research Aim}

\textit{The aim of this research is to design, develop and evaluate a low overhead Kubernetes framework to collect, store and process telemetry data using deep learning to help system operators detect anomalies earlier in order to reduce the \ac{mttr} when the system is experiencing an anomaly.}

In this project, the author tries to create a single model that can monitor all the vitals of a given service and output an anomaly score in any given time window. The author hopes to make it general enough so that operators can take the same model and deploy it with other services, and the model will be adopted to the new services using a few-shot learning method \citep{wang2020generalizing}. To make it happen, the author is trying to create a data encoding technique to represent monitoring data in a programming language/framework-independent way. To achieve this goal the author is also hoping to create a lightweight service instrumentation pipeline that can collect and process telemetry data in real-time without requiring any additional work from the user's end.