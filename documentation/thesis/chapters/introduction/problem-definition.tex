\section{Problem Definition}

One of the leading problems in monitoring microservices is the sheer number of data that they generate. It's humanly impossible to monitor the metrics of all the services and it's hard for a single person to understand the entire system. By\acp{sres} using abstracted metrics called \acp{sli} which measures the quality of the service at a higher level can overcome this. Although \acp{sli} can alert when there is an issue in the system, it is difficult to understand where the actual problem is from this point on. To understand the root cause of the problem, \acp{sres} need to dig into \acp{apm} of all the services and go through each and every log of the troubling services.

When the system consists of hundreds or thousands of services that are interdependent; It is really hard to find where the actual issue is coming from and it may require the attention from all the service owners of the failing services to go through the logs and \acp{apm} to identify the actual root cause of the failure. This could greatly increase the \ac{mttr} and waste a lot of developer time by browsing logs.

\subsection{Problem Statement}

Modern distributed systems are becoming large and complex to the point where, when a failure occurs, it requires collaboration with a large number of people to find the actual root cause. Implementing a \textbf{framework} to make it easier to integrate machine learning models to detect anomalies in real-time could deflate \ac{mttr}.