\section{Contribution to the Body of Knowledge}

\subsection{Technical Contribution}

\begin{itemize}[noitemsep,nolistsep]
    \item \textbf{\href{https://github.com/MrSupiri/MicroSim}{MicroSim}} - A tool to create a simulated distributed system within a Kubernetes Cluster. This tool also consists of its own native load generate which can create semi-random traffic patterns.
    \item \textbf{Lazy-Koala} - A Kubernetes native toolkit that provides a flexible API for collecting telemetry with zero instrumentation. This also helps with integrating any anomaly detection model with the system with minimal effort.
\end{itemize} 

\subsection{Domain Contribution}

\begin{itemize}[noitemsep,nolistsep]
    \item \textbf{Data Encoding Technique} - This project introduced a novel data encoding technique that uses RGB color spectrum to represent telemetry data from different sampling windows. This helps both humans and machine learning models to understand the status of services with a glance.
    \item \textbf{Convolutional Neural Network} - To accompany the encoded data, a convolutional neural network was developed to detect anomalies. This model was trained on a large dataset of telemetry data collected from different services built on a different framework. The helped to make the model very generalizable and easy to fine-tune per-service basis.
\end{itemize} 