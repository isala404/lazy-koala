\section{Use Case Diagram}

\begin{figure}[H]
    \includegraphics[width=15cm]{assets/requirement-specification/use-case.png}
    \caption{Use Case diagram (self-composed)}
    % \label{fig:poc-autoencoder}
\end{figure}


\newcommand{\UseCaseDescription}[9]{
    \textbf{}
    \begin{longtable}{|p{40mm}|p{113mm}|}
    \hline
        \textbf{Use Case ID} & \textbf{#1} \\ \hline
        \textbf{Use Case Name} & #2 \\ \hline
        \textbf{Description} & #3 \\ \hline
        \textbf{Participating actors} & #4 \\ \hline
        \textbf{Preconditions} & #5 \\ \hline
        \textbf{Extended use cases} & #6 \\ \hline
        \textbf{Included use cases} & #7 \\ \hline
        \textbf{Main flow} & #8 \\ \hline
        \UseCaseDescriptionContinued#9
    % \caption{#2 (Self Composed)}
    \end{longtable}
}

\newcommand{\UseCaseDescriptionContinued}[3]{
    \textbf{Alternative flows} & #1 \\ \hline
    \textbf{Exceptional flows} & #2 \\ \hline
    \textbf{Postconditions} & #3 \\ \hline
}

\newenvironment{CompactItemizes}
{ \vspace{-8mm}\begin{itemize}[leftmargin=*,noitemsep,nolistsep]}
{ \vspace{-7mm}\end{itemize}} 

\newenvironment{CompactEnumerate}
{ \vspace{-8mm}\begin{enumerate}[leftmargin=*,noitemsep,nolistsep]}
{ \vspace{-7mm}\end{enumerate}} 


\section{Use Case Descriptions}
Due to the page limits, only the main use-case description is present here. Please find the refer  of the use-case descriptions in Appendix-\ref{appendix:use-case-description}.

\vspace{-2em}
\UseCaseDescription
{UC-04}
{Check for Root Courses}
{Look at the service topology graph to find out the root course of an issue.}
{Software Engineer\newline
Reliability Engineer}
{\begin{CompactItemizes}
    \item kubectl installed and configured to talk to a Kubernetes cluster.
    \item The Kubernetes cluster has a Lazy Koala operator deployed.
    \item Established port forwarding connection with Lazy Koala operator.
\end{CompactItemizes}}
{N/A}
{Generate Report\newline
Read from the database}
{\begin{CompactEnumerate}
    \item Visit the forwarded port on the local machine.
    \item Open Monitor tab.
    \item Inspect the graph.
\end{CompactEnumerate}}
{{N/A}
{N/A}
{N/A}}


\vspace{-2em}
\UseCaseDescription
{UC-07}
{Extract telemetry}
{Every 5 second Gazer will scrape the metric server}
{System Timer}
{\begin{CompactItemizes}
    \item Gazer is deployed to the cluster
\end{CompactItemizes}}
{N/A}
{Write to the database}
{\begin{CompactEnumerate}
    \item poll\_kube\_api function get invoked.
    \item Gazer looks at the config file and finds out the service it’s responsible for.
    \item Query metric server for each of the service names.
    \item Store it in local memory.
\end{CompactEnumerate}}
{{N/A}
{\textbf{E1}: metric server returns non 200 status code.
\vspace{-4mm}\begin{enumerate}
    \item Retry in the next iteration.
\vspace{-7mm}\end{enumerate}}
{\begin{CompactItemizes}
    \item Updated local memory with recent telemetry data.
\end{CompactItemizes}}}

\vspace{-2em}
\UseCaseDescription
{UC-09}
{Check for Anomalies}
{Check for Anomalies in each of the service}
{System Timer}
{\begin{CompactItemizes}
    \item An Instance of Sherlock is deployed.
\end{CompactItemizes}}
{N/A}
{Write to the database}
{\begin{CompactEnumerate}
    \item check\_anomlies function invoked.
    \item Query the database for telemetry for about the last 5 minutes.
    \item Do a forward pass on the model.
    \item Calculate the reconstruction loss.
    \item Store it in local memory.
\end{CompactEnumerate}}
{{N/A}
{N/A}
{\begin{CompactItemizes}
    \item Updated local memory with current reconstruction loss.
\end{CompactItemizes}}}
